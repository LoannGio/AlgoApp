%%%%%%%%%%%%%%%%%%%%%%%%%%%%%%%%%%%%%%%%%%%%%%%%%%%%%%%%%%%%%%%%%%%%%
% LaTeX Template: Project Titlepage Modified (v 0.1) by rcx
%
% Original Source: http://www.howtotex.com
% Date: February 2014
% 
% This is a title page template which be used for articles & reports.
% 
% This is the modified version of the original Latex template from
% aforementioned website.
% 
%%%%%%%%%%%%%%%%%%%%%%%%%%%%%%%%%%%%%%%%%%%%%%%%%%%%%%%%%%%%%%%%%%%%%%

\documentclass[12pt]{report}
\usepackage[a4paper]{geometry}
\usepackage[myheadings]{fullpage}
\usepackage{fancyhdr}
\usepackage{lastpage}
\usepackage{graphicx, wrapfig, subcaption, setspace, booktabs}
\usepackage[T1]{fontenc}
\usepackage[font=small, labelfont=bf]{caption}
\usepackage{fourier}
\usepackage[protrusion=true, expansion=true]{microtype}
\usepackage[english]{babel}
\usepackage{sectsty}
\usepackage{url}
\usepackage{lmodern}

\newcommand{\HRule}[1]{\rule{\linewidth}{#1}}
\onehalfspacing
\setcounter{tocdepth}{5}
\setcounter{secnumdepth}{5}

%-------------------------------------------------------------------------------
% HEADER & FOOTER
%-------------------------------------------------------------------------------
\pagestyle{fancy}
\fancyhf{}
\setlength\headheight{15pt}
\fancyhead[L]{Algorithmique Appliquée}
\fancyhead[R]{L. Giovannangeli, F. Jacques, J. Narboni}
\fancyfoot[R]{Page \thepage\ of \pageref{LastPage}}
%-------------------------------------------------------------------------------
% TITLE PAGE
%-------------------------------------------------------------------------------

\begin{document}

\title{ \normalsize \textsc{Module}
		\\ [2.0cm]
		\HRule{0.5pt} \\
		\LARGE \textbf{\uppercase{Title}}
		\HRule{2pt} \\ [0.5cm]
		\normalsize \today \vspace*{5\baselineskip}}

\date{}

\author{
		Loann Giovannangeli, Fabien Jacques, Jonathan Narboni}

\maketitle
\tableofcontents
\newpage

%-------------------------------------------------------------------------------
% Section title formatting
\sectionfont{\scshape}
%-------------------------------------------------------------------------------

%-------------------------------------------------------------------------------
% BODY
%-------------------------------------------------------------------------------

\chapter*{Modélisation du Problème}
On modélise le problème ainsi :
\newline
\newline
\section{Entrées}
Soient :
\begin{itemize}
\item $o_i \in O$ ($i \in \{1,..., n\}$) Les $n$ adversaires. On notera $o_i.x$ et $o_i.y$ les coordonnées de $o_i$.
\item $d_i \in D$ ($i \in \{1,..., k\}$) Les $m$ défenseurs.
\item $\theta_{step}$ Pas de discrétisation pour les angles de tirs. On considérera "l'angle $k$" comme étant un angle de mesure $k \times \theta_{step}$ radians. On nommera $K_{max}$ le k maximum tel que $k \times \theta_{step} $<$ 2 \pi$.
\item $pos_{step}$, $X_{max}$, $Y_{max}$, Pas de discrétisation pour les positions sur le terrain, Abscisse  maximale et Ordonnée maximale. On considérera qu'un robot est en position $(x, y) \in (N \times N)$ si il est en position ($x \times pos_{step}$, $y \times pos_{step}$).
\end{itemize}

\section{Fonctions nécessaires}
\begin{itemize}
\item $cadre(x, y, k)$ ($x, y, k \in N$) Fonction qui renvoie vrai si le tir d'un attaquant aux coordonnées $(x, y)$ d'angle $k$ est bien cadré et faux sinon.
\item $interception(x, y, k, x', y')$ Fonction qui renvoie vrai si le tir d'un attaquant en (x, y) d'angle k est arrêté par un défenseur en (x', y') et faux sinon.
\end{itemize}
\space
\space

\section{Graphe} \bigbreak
On crée le graphe G suivant :
\subsection{Sommets attaquants}
$\forall i \in \{1, ..., n\}$, $\forall k \in \{0, ..., K_{max} \}$

Si $cadre(o_i.x, o_i.y, k)$ est vrai : on ajoute un sommet au graphe d'étiquette (i, k) qui représente un tir possible de l'attaquant $o_i$. \bigbreak

\subsection{Sommets défenseurs}
$\forall x \in \{1, ..., X_{max}\}$ $\forall y \in \{1, ..., Y_{max}\}$ 

Pour tout sommet d'étiquette (i, k) tel que $intercepte(o_i.x, o_i.y, k, x, y)$ : On crée un sommet d'étiquette (x, y) qui représente un défenseur placé en (x, y) si ce sommet n'existe pas déjà et on ajoute une arête entre $(i, k)$ et $(x, y)$ qui représente le fait qu'un défenseur placé en $(x, y)$ arrêterai le tir $(i, k)$.

\section{Solution du problème}
Une solution du problème est un ensemble de taille minimal de sommets défenseurs qui domine l'ensemble des sommets attaquants.
Pour trouver cet ensemble on peut essayer avec 1, puis 2, puis 3, etc. On sait qu'on pourra toujours trouver un ensemble de taille n (nombre d'attaquants) car un défenseur est capable de bloquer tous les tirs d'un attaquant en se plaçant suffisamment proche de lui. 

\section{Extensions}
\subsection{Gestion des collisions}
Si on veut éviter d'avoir des robots qui rentrent en collision, on peut ajouter une arête entre chaque paire de sommets défenseurs pour lesquels avoir un défenseur sur chacune des positions causerai une collision. Une solution du problème qui évite les collisions doit alors avoir comme condition supplémentaire d'être un ensemble de sommets stable.


%-------------------------------------------------------------------------------
% REFERENCES
%-------------------------------------------------------------------------------
\newpage
\section*{References}
\addcontentsline{toc}{section}{References}

Anand, U., 2010. The Elusive Free Radicals, \textit{The Clinical Chemist,} [e-journal] Available at:<\url{http://www.clinchem.org/content/56/10/1649.full.pdf}> [Accessed 2 November 2013]
\newline
\newline

\end{document}

%-------------------------------------------------------------------------------
% SNIPPETS
%-------------------------------------------------------------------------------

%\begin{figure}[!ht]
%	\centering
%	\includegraphics[width=0.8\textwidth]{file_name}
%	\caption{}
%	\centering
%	\label{label:file_name}
%\end{figure}

%\begin{figure}[!ht]
%	\centering
%	\includegraphics[width=0.8\textwidth]{graph}
%	\caption{Blood pressure ranges and associated level of hypertension (American Heart Association, 2013).}
%	\centering
%	\label{label:graph}
%\end{figure}

%\begin{wrapfigure}{r}{0.30\textwidth}
%	\vspace{-40pt}
%	\begin{center}
%		\includegraphics[width=0.29\textwidth]{file_name}
%	\end{center}
%	\vspace{-20pt}
%	\caption{}
%	\label{label:file_name}
%\end{wrapfigure}

%\begin{wrapfigure}{r}{0.45\textwidth}
%	\begin{center}
%		\includegraphics[width=0.29\textwidth]{manometer}
%	\end{center}
%	\caption{Aneroid sphygmomanometer with stethoscope (Medicalexpo, 2012).}
%	\label{label:manometer}
%\end{wrapfigure}

%\begin{table}[!ht]\footnotesize
%	\centering
%	\begin{tabular}{cccccc}
%	\toprule
%	\multicolumn{2}{c} {Pearson's correlation test} & \multicolumn{4}{c} {Independent t-test} \\
%	\midrule	
%	\multicolumn{2}{c} {Gender} & \multicolumn{2}{c} {Activity level} & \multicolumn{2}{c} {Gender} \\
%	\midrule
%	Males & Females & 1st level & 6th level & Males & Females \\
%	\midrule
%	\multicolumn{2}{c} {BMI vs. SP} & \multicolumn{2}{c} {Systolic pressure} & \multicolumn{2}{c} {Systolic Pressure} \\
%	\multicolumn{2}{c} {BMI vs. DP} & \multicolumn{2}{c} {Diastolic pressure} & \multicolumn{2}{c} {Diastolic pressure} \\
%	\multicolumn{2}{c} {BMI vs. MAP} & \multicolumn{2}{c} {MAP} & \multicolumn{2}{c} {MAP} \\
%	\multicolumn{2}{c} {W:H ratio vs. SP} & \multicolumn{2}{c} {BMI} & \multicolumn{2}{c} {BMI} \\
%	\multicolumn{2}{c} {W:H ratio vs. DP} & \multicolumn{2}{c} {W:H ratio} & \multicolumn{2}{c} {W:H ratio} \\
%	\multicolumn{2}{c} {W:H ratio vs. MAP} & \multicolumn{2}{c} {\% Body fat} & \multicolumn{2}{c} {\% Body fat} \\
%	\multicolumn{2}{c} {} & \multicolumn{2}{c} {Height} & \multicolumn{2}{c} {Height} \\
%	\multicolumn{2}{c} {} & \multicolumn{2}{c} {Weight} & \multicolumn{2}{c} {Weight} \\
%	\multicolumn{2}{c} {} & \multicolumn{2}{c} {Heart rate} & \multicolumn{2}{c} {Heart rate} \\
%	\bottomrule
%	\end{tabular}
%	\caption{Parameters that were analysed and related statistical test performed for current study. BMI - body mass index; SP - systolic pressure; DP - diastolic pressure; MAP - mean arterial pressure; W:H ratio - waist to hip ratio.}
%	\label{label:tests}
%\end{table}
